\documentclass[12pt]{article}

\usepackage{amsmath}
\usepackage{amssymb}
\usepackage{geometry}
\usepackage{graphicx}
\usepackage{hyperref}

\geometry{letterpaper,tmargin=1in,bmargin=1in,lmargin=1in,rmargin=1in}

\hypersetup{
colorlinks, linkcolor=blue,
}


\begin{document}

\title{Analog Electronics}
\author{Laboratory exercise 2}
\date{Fall 2016}
\maketitle

\newpage
\section{Abstract}

In this experimentation we will receive a mystery amplifier from which we will determine its gain, $R_i$, and its $R_o$; using our oscilloscopes in combination with out function generator. Once those variables are established we will also plot a 15 point graph showing the linearity of the amplifier and its reaction to various amplitudes. Further more we will form up with another group to create a cascading amplifier that gives us a gain of 60 on our load 2.2k resistor.

\section{Theory}

By being able to calculate and measure our mystery static amplifier we will be able to create a cascading system that suits our needs. In some cases we may be unable to produce a amplifier with a gain of 60, but by using various amplifiers we can reproduce a gain of 60. The same way that there is no such thing as a $105k \Omega$, we can create one through the use of a $100k \Omega$ and a $5k \Omega$ in series.


\newpage

\section{Experimentation}
\subsection{Experiment setup}




We begin the experiment by receiving our mystery circuit, Amplifier $D$. The first thing will do is calculate the gain of $D$. \\

To power our amplifier we will utilize the terminals atop of breadboard.
\begin{enumerate}
	\item Starting from the left, The first terminal will be connected to $+12v$, the second to $-12v$, and the third terminal to ground.
\end{enumerate}

Next we will setup our function generator to power our circuit.

\begin{enumerate}
	\item Power on the function generator
	\item Set the frequency of our sin wave to $1kHz$
	\item Set the amplitude of our sine wave to around $300mV$
	\item Plug our termination into the function generator output 1, followed by our wire that will be attached to our circuit.
	\item Take the ground from the function generator and plug it into the BLUE vertical connection to the far left of the board.
	\item Take the positive from the function generator and plug it into the RED vertical connection to the far left of the board.
\end{enumerate}

Now we will setup our oscilloscope to compare our $V_i$ to our $V_o$.

\begin{enumerate}
	\item Power on the oscilloscope
	\item Ensuring that probes are attached to input 1 on the oscilloscope, attach the ground clip to the BLUE vertical ground on the far RIGHT of the board
	\item Attach the positive probe to the far right of the board on the RED vertical connection.
\end{enumerate}

Finally we will power our amplifier on.

\begin{enumerate}
	\item Power on channel one on both the oscilloscope and the function generator.
	\item Press "Autoset" on your function generator. 
	\item The output sinusoid will appear.
\end{enumerate}
\textit{Note: If your output sinusoid is clipping, adjust your input amplitude to a lower value.}

\newpage


\subsection{Measurements}
To calculate our gain we will compare our $V_i$ from our function generator to our $V_o$ on our oscilloscope.

\begin{enumerate}
	\item Find an input amplitude value that ensures our output sinusoid is not clipping. This value will be our $V_i$
	\item Using the occiliscope measure the pk-pk voltage of the output sinusoid, this will be our $V_o$
\end{enumerate}
From these two values we can calculate the gain,$A_v,$ through our amplifiers with the equation

$$A_v = \frac{V_o}{V_i}$$

When we plug our values in we get

$$A_v = \frac{V_o}{V_i} = \frac{100mV}{3.13V} = 31$$

The gain for amplifier $D$ is $\approx 30$\\

From the various methods of measurements we used, such as placing and finding the current across a resistor we concluded that our input $R_o$ is near zero.\\

From by adjusting out function generator amplitude from $0-700mV$ by intervals of $50$ we will create a table, then graph its linearity.

\begin{table}[h]
	\centering
	\caption{$V_i$ vs $V_i$}
	\label{my-label}
	\begin{tabular}{lll}
		$V_i$ mV & $V_o$ & Gain        \\
		700          & 22    & 31.81818182 \\
		650          & 20.5  & 31.70731707 \\
		600          & 19    & 31.57894737 \\
		550          & 17.3  & 31.79190751 \\
		500          & 15.8  & 31.64556962 \\
		450          & 13.9  & 32.37410072 \\
		400          & 12.4  & 32.25806452 \\
		350          & 10.8  & 32.40740741 \\
		300          & 9.38  & 31.98294243 \\
		250          & 7.75  & 32.25806452 \\
		200          & 6.23  & 32.10272873 \\
		150          & 4.74  & 31.64556962 \\
		100          & 3.12  & 32.05128205 \\
		50           & 1.62  & 30.86419753
	\end{tabular}
\end{table}
\newpage
\subsubsection{Graphing $V_i$ and $V_o$}	
From the table we created we can graph the relationship between our inputs and outputs to better illustrate our op-amp

\begin{figure}[!h]
	\centering
	\includegraphics[width=0.8\textwidth]{graph1.png}
	\label{fig:lab2_graph}
\end{figure}

\subsubsection{Calculating Cut-off frequency}

To calculate our cutoff frequency we will use our function generator to send a sinusoid with an amplitude that is too great for our amplifier to power. This will result in a near square wave. The value that we arrived at was $1200KHz$

\subsection{Cascading Amplifiers}
For the next part of out experimentation we will combine teams with another mystery amplifier to arrive at a gain of 60 across a $2.2K\Omega$ resistor. When calculating gain we simply multiply the gain of one amplifier to another. In our case an amplifier of 2 in combination with our 30 will arrive at a gain of 60. By combining our amplifier in series with group A we were able to arrive at a gain of 60 across a $100K\Omega$ resistor. \\

We had an initial problem where we could not get the full voltage to our $2.2K\Omega$ resistor. To solve this problem we created a much larger load to ensure that all the voltage arrived at it rather than the $940\Omega R_0$. We where later corrected and shown that a gain of 60 can be achieved with the original load resistor


\newpage


\section{Conclusion}

In this experimentation we used a variety of analysis techniques to calculate several variables for our mystery amplifier. From this point we measured numerous data points from various input amplitudes and created a linear graph of the data. By combining with another mystery amplifier we where able to successfully create a gain of 60.


\end{document}
