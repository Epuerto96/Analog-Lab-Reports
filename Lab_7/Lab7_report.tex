\documentclass[12pt]{article}

\usepackage{amsmath}
\usepackage{amssymb}
\usepackage{geometry}
\usepackage{graphicx}
\usepackage{hyperref}

\geometry{letterpaper,tmargin=1in,bmargin=1in,lmargin=1in,rmargin=1in}

\hypersetup{
colorlinks, linkcolor=blue,
}


\begin{document}

\title{Analog Electronics}
\author{I-V Characteristics of N-Type Power Mosfets}
\date{Fall 2016}
\maketitle

\newpage
\section{Abstract}

In this experimentation we will measure the I-V characteristics of a N-type power mosfet. Through the use of several circuit analysis techniques we will discover the voltage threshold $V_t$ and using it we will plot lines for several threshold voltages. This will allow for the characteristic of the transistor to also be plotted by iterating VDS and measuring with respects to $I_D$.
\section{Materials}
\begin{enumerate}
	\item Bread Board
	\item N-Type Power Mosfet
	\item Jumper Cables
	\item DC Voltage Source
	\item Multimeter
	\item Various Probes
\end{enumerate}


\section{Finding $V_t$}
\begin{enumerate}
	\item Begin by wiring the mosfet and ensuring that voltage is indeed being applied to the gate
	\item Using the multimeter measure the current $I_d$
	\item Increase the Voltage $V_{gs}$ until current begins to flow.
	\item Once current begins to flow, note the voltage $V_{gs}$ that is the $V_t$
\end{enumerate}

\section{I-V Characteristics}
To plot the I-V characteristics graph we must first gather the necessary values.
\begin{enumerate}
	\item Using several values of $V{gs}$ that are at or above $V_{t}$ begin to change the variable resistance $R_1$.
	\item While changing $V_{ds}$ make note of its value from 0-10 while also keeping track of the values of $I_d$
	\item The resulting data is the points that will be used to plot your data.
\end{enumerate}

\begin{table}[]
	\centering
	\caption{My caption}
	\label{my-label}
	\begin{tabular}{llllll}
		VGS at 2.4v &        & VGS at 2.6 &        & VGS at 2.2 &        \\
		VDS         & ID mA  & VDSmV      & IDmA   & VDSmV      & IDmA   \\
		1.4         & 0.002  & 0.2        & 0.001  & 4.9        & 0.0008 \\
		2.4         & 0.0036 & 0.3        & 0.0032 & 7          & 0.001  \\
		3.5         & 0.0051 & 0.4        & 0.0052 & 10.8       & 0.0014 \\
		4.4         & 0.0067 & 0.5        & 0.0064 & 12.6       & 0.0015 \\
		5.3         & 0.0076 & 0.7        & 0.01   & 16.1       & 0.0018 \\
		7           & 0.01   & 1.25       & 0.0196 & 22.8       & 0.0023 \\
		8.5         & 0.012  & 1.85       & 0.03   & 29.6       & 0.0028 \\
		10          & 0.014  & 2.5        & 0.041  & 37.4       & 0.003  \\
		12.3        & 0.0165 & 3.2        & 0.0535 & 43         & 0.003  \\
		18          & 0.02   & 4.1        & 0.0688 & 68.9       & 0.0039 \\
		38          & 0.0344 & 6.3        & 0.0956 & 101.5      & 0.0043 \\
		52.4        & 0.0396 & 8.2        & 0.1311 & 197.5      & 0.0043 \\
		78.5        & 0.044  & 9.7        & 0.152  & 571        & 0.0048 \\
		96.8        & 0.0457 & 11.3       & 0.1745 & 1596       & 0.0052 \\
		150         & 0.0477 & 20.5       & 0.2837 & 3076       & 0.0057 \\
		342         & 0.054  & 33.6       & 0.3972 & 5028       & 0.0068 \\
		960         & 0.0545 & 43.9       & 0.457  &            &        \\
		2.5         & 0.061  & 54.9       & 0.5047 &            &        \\
		3           & 0.0636 & 70.7       & 0.548  &            &        \\
		4.65        & 0.0707 & 100        & 0.5945 &            &        \\
		5           & 0.0718 & 213        & 0.649  &            &       
	\end{tabular}
\end{table}

\section{Conclusion}

In this experimentation we used a variety of analysis techniques to calculate and design a transistor circuit that was then manipulated in a variety of ways to test for $V_t$ and also to find the I-V Characteristics. One of the major problems was my lack of experience in excel. In the future I will try my best to plot the data in a was that is actually coherent.
\end{document}